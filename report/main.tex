\documentclass{Assignment}

\usepackage[style=authoryear-ibid,backend=biber]{biblatex}
\usepackage{amsfonts}

\CourseCode{COMS3005A}
\CourseName{Advanced Analysis of Algorithms}
\AssignmentNumber{Phase 1}
\AssignmentName{}
\DueDate{2022-04-18}
\StudentNumber{2426285}
\StudentName{Brendan Griffiths}

\addbibresource{main.bib}

\begin{document}
    \maketitle
    \tableofcontents

    \section*{Plagiarism Declaration}
        I, Brendan Griffiths (2426285) am a student registered for BSc in Computer Science in the year
        2023. I hereby declare the following:
        \begin{itemize}
            \item I am aware that plagiarism (the use of someone else's work without their permission and/or without acknowledging the original source) is wrong.
            \item I confirm that ALL the work submitted for assessment for the above course is my own unaided work except where I have explicitly indicated otherwise.
            \item I have followed the required conventions in referencing the thoughts and ideas of others.
            \item I understand that the University of the Witwatersrand may take disciplinary action against me if there is a belief that this is not my own unaided work that I have failed to acknowledge the source of the ideas or words in my writing.
        \end{itemize}

    \section{Aim}
        The identification and usage of adjacencies between rectangles is a common problem in Computer Science.
        <<not finished>>
    
    \section{Data Generation}
        \paragraph{
            The two methods of data generation used for testing the algorithm are a subdivision process and rectangle appending. Each process is explored below.
        }
        \subsection{Appending Rectangles}
            \paragraph{ This methodology of data generation creates complex tree-like structures, with no guarantee that any rectangle, except the first, has an adjacency. This offers advantages by creating unique data sets that can feature two rectangles sharing a left adjacency with another rectangle, space between rectangles, and other similar elements. It does however have the disadvantage that a newly generated rectangle is unlikely to 'generate' behind a previously created rectangle.}

            \paragraph{The following figures demonstrate data generated by this method.}
            \begin{itemize}
                \item Figure example
                \item Figure example
            \end{itemize}

            \paragraph{The following algorithm was used to create the data:}
            \begin{enumerate}
                \item Create a rectangle with random dimensions (width and height) where the rectangle is constrained at the origin.
                \item Add the origin rectangle to your list
                \item Select a random rectangle from the list
                \item Create a new rectangle with random dimensions where the rectangle is constrained to the selected rectangles right border
                    \begin{enumerate}
                        \item Set the new rectangles left x value to the right x value of the selected rectangle
                        \item Set the new rectangles bottom y value as a random point between the selected rectangle's bottom y and top y values.
                    \end{enumerate}
                \item Validate that the newly generated rectangle does not overlap with any other rectangle
                    \begin{enumerate}
                        \item Loop through each already existing rectangle
                        \item Check that the new rectangles boundaries do not fall within all the other rectangles boundaries
                    \end{enumerate}
                \item If step 5 is true, then add the rectangle to the list, else do not add the rectangle to the list
                \item If the required number of rectangles exist in the list, then stop, else go to step 3
            \end{enumerate}
                
        \subsection{Subdivision}
            \paragraph{
                <<Not Done Yet>>
            }
    
    \section{Algorithm}
        \begin{enumerate}
            \item Loop through each rectangle in the list as $r1$
            \item Loop through each other rectangle in the list as $r2$
            \item If $r2$ is vertically adjacent to $r1$, add $r2$ to $r1$'s list of adjacencies
                \begin{enumerate}
                    \item $r2$ is vertically adjacent to $r1$ if $r1$'s right $x$ value is equal to $r2$'s left $x$ value and either of $r2$'s $y$ values fall between $r1$'s $y$ values.
                \end{enumerate}
        \end{enumerate}
    
    \section{Theoretical Analysis}
    \section{Implementation Details}
    \section{Results}
    \section{Conclusion}
    \printbibliography % Empty because I don't cite anything. Once I cite something it'll load correctly
    \section*{Appendix}

    % Citation Example
    % \cite{knuth-fa,knuth-acp}
\end{document}
